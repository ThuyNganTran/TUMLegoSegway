
% Alle Konfigurationsbefehle sind optional. Fehlende Befehle fueheren einfach
% zu "blank forms".

% Typ der Arbeit/Einstellung. Gueltige Argumente sind:
% bachelor,master,diplom,idp,gr,hiwi,other
% Falls 'other' gewaehlt wird, kann als optionales Argument eine spezielle Art
% von Abschlussarbeit angegeben werden, z.B. \type[Sklave]{other}. Andernfalls
% wird 'Other' als Standardbeschreibung gesetzt.
\type{bachelor}

% Informationen ueber den Studenten. Sollte selbsterklaerend sein.
\anrede{Herr}
\nachname{nachname}
\vorname{vorname}
\matrikel{matrikel}
\sunhalle{sunaccount}
\semester{1}{SoSe\,2016}
\studientelefon{}{tel}
\heimattelefon{}{--}
\studienadresse{strasse}{plz stadt}
\heimatadresse[adresszusatz=,appartment=]{}{}
\mail{student@tum.de}

% Informationen ueber die Arbeit. Sollte selbsterklaerend sein.
\themensteller{\NEThead}
\beginn{04}{2016}
\endt{08}{2016}
\betreuer{Stephan G\"unther, Maurice Leclaire}
\title{Development of a Demonstrator for a Wireless Cyber-Physical Network}{Entwicklung eines Demonstrators für ein drahtloses cyber-physisches Netzwerk}
\studiengang{Informatik}


% Falls \type{hiwi} gesetzt wurde, wird die Taetigkeit auf dem Aufnahmeformular
% des Lehrstuhls angegeben.
\taetigkeit{test}

